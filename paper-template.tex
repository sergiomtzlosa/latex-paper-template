% !TeX encoding = UTF-8 Unicode
% !TeX TS-program = pdflatex
% !TeX root = paper-template.tex
% !TeX spellcheck = en-GB
% !BIB program = bibtex
\documentclass[10pt, conference, compsocconf]{classes/IEEEtran}

\usepackage[OT1]{fontenc}
%\usepackage[T1]{fontenc}
\usepackage[utf8]{inputenc} % accents and UTF-8 encoding 
\usepackage[main=english,spanish]{babel}
\usepackage[none]{hyphenat} % do not split words at the en of line
%\usepackage{parskip} % space between paragraphs
\usepackage[cmex10]{amsmath} % It does not apply typewrtter style on formulas
\usepackage[protrusion=true,expansion,final]{microtype} % https://tex.stackexchange.com/questions/37029/text-out-of-margins	
\usepackage{graphicx}
\usepackage{eso-pic}
\usepackage{algorithmic}
\usepackage{array}
\usepackage{mdwmath}
\usepackage{mdwtab}
\usepackage{eqparbox}
\usepackage{./extensions/support-caption}
\usepackage[labelfont=bf]{caption}
\usepackage[caption=false,font=footnotesize]{subfig}
\usepackage{pdftexcmds}
\usepackage{catchfile}
\usepackage{placeins}
\usepackage{./extensions/bibspacing}
\usepackage[superscript,biblabel]{cite}
\usepackage{microtype}
%\usepackage{url}
\usepackage{hyperref}
\usepackage{lipsum} 

\graphicspath{{./images/}}
\DeclareGraphicsExtensions{.pdf,.jpeg,.png}

\definecolor{blue-hyperref}{rgb}{0,0.2,0.6}

\hypersetup{
	colorlinks = true,
	linkcolor=blue-hyperref,
	citecolor=blue-hyperref,
	filecolor=blue-hyperref,
	urlcolor=blue-hyperref,
	runcolor=blue-hyperref,
	menucolor=blue-hyperref,
	linkbordercolor=blue-hyperref,
	citebordercolor=blue-hyperref,
	filebordercolor=blue-hyperref,
	urlbordercolor=blue-hyperref,
	runbordercolor=blue-hyperref,
	menubordercolor=blue-hyperref,
	pdfpagemode=UseOutlines,
	hypertexnames = true,
	pdfencoding = auto, 
	psdextra, 
	bookmarksdepth = 4
	bookmarks=true,
	bookmarksopen=true,    
	bookmarksopenlevel=0,
	bookmarksnumbered=true,
	plainpages=false
}
\setlength{\marginparwidth}{2cm}

\def\spaceBelowCaptionFigures{-10pt}

% reformat caption indentation and alignment
% https://tex.stackexchange.com/questions/275131/align-caption-to-the-left
\newcommand{\centercaption}[1]{
	\captionsetup{belowskip=\spaceBelowCaptionFigures,justification=raggedright, singlelinecheck=true}
	\protect\caption{#1}
}

\newcommand{\insertimage}[4][0.5]{
	
	\begin{figure}[ht]
		\centering
		\includegraphics[width=#1\textwidth,height=\textheight,keepaspectratio]{#2}
		\captionsetup{belowskip=\spaceBelowCaptionFigures, justification=raggedright, singlelinecheck=true}
		\centercaption{#3}
		\ifthenelse{\equal{#4}{}}{}{\label{#4}}
	\end{figure}
%	\FloatBarrier
}

% brakets for citation superscripts
% https://tex.stackexchange.com/questions/79591/superscripts-in-bibliography-with-bibtex#79599
\makeatletter 
\renewcommand{\@citess}[1]{\textsuperscript{[#1]}} 
\makeatother

% page number in first page
\makeatletter
\g@addto@macro\maketitle{\thispagestyle{plain}}  
\makeatother

\setlength{\bibitemsep}{.5\baselineskip plus .05\baselineskip minus .05\baselineskip}

% Supress warnings
\usepackage[immediate]{silence} % Silence warnings package

\WarningFilter{latex}{Marginpar on page}
\WarningFilter{facyhdr}{\headheight is too small}
\WarningFilter[pdftoc]{hyperref}{Token not allowed in a PDF string}
\WarningFilter{latex}{Underfull*}
\WarningFilter{latex}{Overfull*}
\WarningFilter{latex}{Citation}
\WarningFilter{hyperref}{name}
\WarningFilter{pdftex}{destination with the same} % Filter duplicated ref locations
\WarningFilter{hyperref}{Token not allowed in a PDF string} % Filter warning to silence
\WarningFilter{hyperref}{destination with the same identifier}
% Filter to avoid bad numbering on refs
\WarningFilter{latexfont}{Size substitutions with differences}
% Filter font load for text
\WarningFilter{latex}{`h' float specifier changed to `ht'}  % Filter unwanted positions
\WarningFilter{latexfont}{Font shape} % Filter font double-load
\WarningFilter{cmap}{fontenc already loaded - some fonts may be unprocessed.} % Filter
\WarningFilter{LaTeX}{You have requested package*}
\WarningFilter{TeX}{destination with the same identifier*}
\WarningFilter{LaTeX}{You have requested package*}
\WarningFilter{LaTeX}{Command*}
\WarningFilter[pdftoc]{hyperref}{Token not allowed in a PDF string}
\WarningFilter[pdfTeX]{destination with the same identifier*}

% Custom warning filter
% https://tex.stackexchange.com/questions/223199/suppress-pdflatex-warning-about-missing-eps-file
%\WarningFilter[draft-mode-warning]{pdfTeX}{pdfdraftmode enabled}
%
%\makeatletter
%
%\let\sl@warning\@warning
%\def\@warning#1{%
%	\def\sl@PackageName{pdfTeX}%
%	\ifsl@NoLine
%		\sl@NoLinefalse
%	\else
%		\sl@StoreMessage{#1}%
%	\fi
%	\sl@warning{#1}}
%
%\def\@warning@no@line#1{%
%	\sl@StoreMessage{#1}%
%	\sl@NoLinetrue
%	\@warning{#1\@gobble}}
%
%\makeatother
%
%\ActivateWarningFilters[draft-mode-warning]

\WarningsOff* % Filter other warnings

% Supress warnings
\hbadness=10000 % A parameter that tells TeX at what point to report badness errors (i.e. underfull and overfull error). [number] ranges from 0 to 10000. 
\vbadness=10000 % A parameter that tells TeX at what point to report badness errors (i.e. underfull and overfull error). [number] ranges from 0 to 10000. 
\hfuzz=3000pt % A parameter that allows hbox's to be overfull by [length] before an overfull error occurs. 
\protect\pretolerance=10000
\protect\tolerance=10000
%\overfullrule=1000mm

\protect\tolerance=10000
\protect\hbadness=10000

\pdfstringdefDisableCommands{
	\def\\{}
	\def\texttt#1{<#1>}
	\def\medskip{}
	\def\smallskip{}
	\def\vspace{}
}

\newcount\hbadness
\newdimen\hfuzz

\vfuzz=30pt % https://tex.stackexchange.com/questions/64459/overfull-vbox-warning-disable

%\setlength{\emergencystretch}{3em}


\begin{document}

\title{My paper title}

\author{\IEEEauthorblockN{
    John Doe\IEEEauthorrefmark{1},
    Michael Doe\IEEEauthorrefmark{2}
    \IEEEauthorblockA{\IEEEauthorrefmark{1}Email:  johndoe@johndoe.com}
    \IEEEauthorblockA{\IEEEauthorrefmark{2}Email:  michaeldoe@michaeldoe.com}
  }
}

% use for special paper notices
%\IEEEspecialpapernotice{(Invited Paper)

\pagenumbering{arabic}

\thispagestyle{plain}
\pagestyle{plain}

% make the title area
\maketitle

\begin{abstract}
  \lipsum[1]
\end{abstract}

\begin{IEEEkeywords}
key1, key2, key3, key4, key5, key6, key7, key8, key9, key10
\end{IEEEkeywords}

% For peer review papers, you can put extra information on the cover
% page as needed:
% \ifCLASSOPTIONpeerreview
% \begin{center} \bfseries EDICS Category: 3-BBND \end{center}
% \fi
%
% For peerreview papers, this IEEEtran command inserts a page break and
% creates the second title. It will be ignored for other modes.
\IEEEpeerreviewmaketitle

\section{Introduction}

\lipsum[2-4]
\section{Big explanations}

\lipsum[2-2]\cite{Dinh2015}

\subsection{More explanations}

\lipsum[2-4]\cite{Andrusenko2005}
\section{Demo for a Proof of Concept}

\lipsum[2-3]

\subsection{Demo hardware}

\lipsum[2-5]
\section{Conclusion and future work}

\lipsum[2-4]


% use section* for acknowledgement
%\section*{Acknowledgment}

% The authors would like to thank...
% more thanks here

% An example of a floating figure using the graphicx package.
% Note that \label must occur AFTER (or within) \caption.
% For figures, \caption should occur after the \includegraphics.
% Note that IEEEtran v1.7 and later has special internal code that
% is designed to preserve the operation of \label within \caption
% even when the captionsoff option is in effect. However, because
% of issues like this, it may be the safest practice to put all your
% \label just after \caption rather than within \caption{}.
%
% Reminder: the "draftcls" or "draftclsnofoot", not "draft", class
% option should be used if it is desired that the figures are to be
% displayed while in draft mode.
%
%\begin{figure}[!t]
%\centering
%\includegraphics[width=2.5in]{myfigure}
% where an .eps filename suffix will be assumed under latex, 
% and a .pdf suffix will be assumed for pdflatex; or what has been declared
% via \DeclareGraphicsExtensions.
%\caption{Simulation Results}
%\label{fig_sim}
%\end{figure}

% Note that IEEE typically puts floats only at the top, even when this
% results in a large percentage of a column being occupied by floats.


% An example of a double column floating figure using two subfigures.
% (The subfig.sty package must be loaded for this to work.)
% The subfigure \label commands are set within each subfloat command, the
% \label for the overall figure must come after \caption.
% \hfil must be used as a separator to get equal spacing.
% The subfigure.sty package works much the same way, except \subfigure is
% used instead of \subfloat.
%
%\begin{figure*}[!t]
%\centerline{\subfloat[Case I]\includegraphics[width=2.5in]{subfigcase1}%
%\label{fig_first_case}}
%\hfil
%\subfloat[Case II]{\includegraphics[width=2.5in]{subfigcase2}%
%\label{fig_second_case}}}
%\caption{Simulation results}
%\label{fig_sim}
%\end{figure*}
%
% Note that often IEEE papers with subfigures do not employ subfigure
% captions (using the optional argument to \subfloat), but instead will
% reference/describe all of them (a), (b), etc., within the main caption.


% An example of a floating table. Note that, for IEEE style tables, the 
% \caption command should come BEFORE the table. Table text will default to
% \footnotesize as IEEE normally uses this smaller font for tables.
% The \label must come after \caption as always.
%
%\begin{table}[!t]
%% increase table row spacing, adjust to taste
%\renewcommand{\arraystretch}{1.3}
% if using array.sty, it might be a good idea to tweak the value of
% \extrarowheight as needed to properly center the text within the cells
%\caption{An Example of a Table}
%\label{table_example}
%\centering
%% Some packages, such as MDW tools, offer better commands for making tables
%% than the plain LaTeX2e tabular which is used here.
%\begin{tabular}{|c||c|}
%\hline
%One & Two\\
%\hline
%Three & Four\\
%\hline
%\end{tabular}
%\end{table}


% Note that IEEE does not put floats in the very first column - or typically
% anywhere on the first page for that matter. Also, in-text middle ("here")
% positioning is not used. Most IEEE journals/conferences use top floats
% exclusively. Note that, LaTeX2e, unlike IEEE journals/conferences, places
% footnotes above bottom floats. This can be corrected via the \fnbelowfloat
% command of the stfloats package.


% trigger a \newpage just before the given reference
% number - used to balance the columns on the last page
% adjust value as needed - may need to be readjusted if
% the document is modified later
%\IEEEtriggeratref{8}
% The "triggered" command can be changed if desired:
%\IEEEtriggercmd{\enlargethispage{-5in}}

% references section

% can use a bibliography generated by BibTeX as a .bbl file
% BibTeX documentation can be easily obtained at:
% http://www.ctan.org/tex-archive/biblio/bibtex/contrib/doc/
% The IEEEtran BibTeX style support page is at:
% http://www.michaelshell.org/tex/ieeetran/bibtex/
%\bibliographystyle{IEEEtran}
% argument is your BibTeX string definitions and bibliography database(s)
%\bibliography{IEEEabrv,../bib/paper}
%
% <OR> manually copy in the resultant .bbl file
% set second argument of \begin to the number of references
% (used to reserve space for the reference number labels box)
%\begin{thebibliography}{1}


%\bibitem{IEEEhowto:kopka}
%H.~Kopka and P.~W. Daly, \emph{A Guide to \LaTeX}, 3rd~ed.\hskip 1em plus
%  0.5em minus 0.4em\relax Harlow, England: Addison-Wesley, 1999.

%\end{thebibliography}

\bibliographystyle{classes/IEEEbib}
\bibliography{paper-template}

\end{document}
